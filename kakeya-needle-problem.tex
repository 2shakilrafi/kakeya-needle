\documentclass{article}
\usepackage{amssymb}
\usepackage{url}
\begin{document}
	\title{Proposal for a booth of the Kakeya Needle Conjecture}
	\author{Shakil Rafi}
	\maketitle
	
	\section{The Problem}
	Let $I \subseteq \mathbb{R}$ be a unit interval, which practically speaking is a needle in our booth. Given a connected  set $K \subseteq \mathbb{R}^d$, which in our demonstrated could be printed shape on a piece of letter size paper. Is there a minimum area that shape $K$ can take, and what is it? Propoese by Soichi Kakeya. 
	\\~\\
	For detailed descriptions of the problem, these resources will come in handy:
	\begin{enumerate}
		\item \url{https://math.uchicago.edu/~may/REU2021/REUPapers/Fox.pdf}
		\item \url{https://en.wikipedia.org/wiki/Kakeya_set#Results}
		\item \url{https://www.youtube.com/watch?v=IM-n9c-ARHU}
		\item \url{https://www.youtube.com/watch?v=j-dce6QmVAQ}
	\end{enumerate}
	\section{Why this problem?}
	There are several reasons we may want to use this problem:
	\begin{enumerate}
		\item This problem is very easy to explain
		\item This requires very little equipment see Section \ref{sec:equip}. 
		\item This is often not a common problem students would be familiar with. Typical problems like four color or bridges of K\"onigsberg are already something advanced high school students may be familiar with.
		\item It is somewhat actively being researched unlike the four color or bridges of K\"onigsberg problem. 
	\end{enumerate}
	\section{Equipment Needed}\label{sec:equip}
	We need two equipments:
	\begin{enumerate}
		\item Several pieces of letter size paper with increasingly smaller area Kakeya sets, multiple copies of each.
		\item A sewing needle. Size 10 ``Sharps'' recommended, although given the age of students something less sharp and larger may be needed. Visibility and size may pose problem for students with accessibility needs. 
		\item (Optional) A poster
	\end{enumerate}
	\section{Engagement pathway for students}
	Here is a step by step proposed pathway for engaging with students
	\begin{enumerate}
		\item Explain the problem to students
		\item Give them the obvious solution a circle
		\item Ask if we canmake it smaller
		\item Give them the deltoid example
		\item Engage by asking if we can make it smaller?
		\item Show them some of the ``spiky'' looking Kakeya sets
		\item (for very enthusiastic students) We may show general construction methods and event talk about Hausdorff or Minkowski dimensions. Or not.
		\item Give the sheets of paper with Kakeya sets away to students, they make cool posters! Or fantastic tattoo ideas!
	\end{enumerate}
\end{document}